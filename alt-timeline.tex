\documentclass[a4paper,landscape]{article}

\usepackage[dvipsnames]{xcolor}
\usepackage{\string ~/mylatexstuff/ccbyandreas}
\usepackage[top=2cm, bottom=2cm, left=2cm, right=2cm]{geometry}
\usepackage{etoolbox}
\usepackage{graphicx}
\usepackage{footnote} % to save and spew notes
\usepackage{ccicons}
\usepackage{calc}
\def\UrlFont{\rmfamily\itshape} % roman font in urls
\usepackage{ragged2e}
\usepackage{tikz}
\usepackage{varwidth} % nodes size with adjusted width
\usetikzlibrary{calc, backgrounds}
\usepackage{polyglossia}
\setmainlanguage{english}
\setotherlanguage{arabic}

\usepackage{fontspec}
\setmainfont[Numbers=OldStyle]{Linux Libertine O}
\setsansfont[Numbers=OldStyle]{Source Sans Pro ExtraLight}

% Constants
\def\startce{600}
\def\fince{1550}

% Color to mix in with dynasty. 70%
\def\mixcolor{gray}

\frenchspacing
	
% Vertical position of beginning of name
\newlength\nameh
\setlength\nameh{6.1cm}

\newlength\dynastyheight
\setlength\dynastyheight{.4cm}
\newlength\dynastyvshift
\setlength\dynastyvshift{.2cm}

\newlength\dynwidth

% \grammarian[horizontal offset][east]{<name>}{<yod c.e.>}{<opus magnus>}
\newcommand\person[4][0]{
    % Name and yod on top
    \path [draw] (#3,\nameh) ++ (#1,0) node [rotate=-90, anchor=west] (name) {\makebox[2.5em][l]{#3}#2};
    \draw [thin] (name.east) |- (#3,1.5) -- (#3,0) ;
    % Work below
    \node [font=\itshape, rotate=-90, anchor=west, yshift=#1] (work) at (#3,-2\baselineskip) {#4};
}

\newcommand\infobox[1]{%
  \draw[gray] let \p1 = (work) in node (infobox)
  [color=gray, anchor=north,draw=black, align=left, font=\scriptsize\sffamily]
   at (\x1,-5) {\begin{varwidth}{2.5cm}\RaggedRight#1\end{varwidth}};
  \draw (infobox) -- (work.east);
}


% \dynasty{<color>}{<name>}{<startyear c.e.>}{<endyear c.e.>}{<graphical level (>0)>}
\newcommand\dynasty[5]{%
\begin{pgfonlayer}{background}
    % bar
    \draw [yshift=\dynastyvshift+#5\dynastyheight, line width=\dynastyheight, color=#1!30!\mixcolor]
    (#3,0) -- (#4,0);
    % label
    \node [white, yshift=\dynastyvshift+#5\dynastyheight, font=\sffamily] 
    at ({#3+(#4-#3)/2},0) {#2};
\end{pgfonlayer}
}


% phases
% \phase{<name>}{<begin>}{<end>}
\newcommand{\phase}[3]{
    \draw [thin] (#2+2, \nameh+2mm) -- (#2+2, \nameh+4mm) -- (#3-2, \nameh+4mm) node [midway, above, align=center, font=\sffamily] {#1} -- (#3-2, \nameh+2mm);
}


\setlength\parindent{0pt}

\begin{document}
\thispagestyle{empty}

\begin{minipage}[t]{.4\textwidth}
{\LARGE Timeline of Arab grammarians\\and their major works}\\[\medskipamount]
\textit{v.1.4\\\today}
\end{minipage}
\hfill
\begin{minipage}[t]{.38\textwidth}
This timeline is based primarily on \textit{The Arabic Linguistic Tradition} by G. Bohas, \mbox{J.-P.} Guillaume, and D. Kouloughli (Georgetown University Press, 2006) and lists grammarians mentioned therein. Years are in the common era.
\end{minipage}


\vfill

\begin{tikzpicture}[x=\textwidth/(\fince-\startce)]

    \node at (650,\nameh-1ex) [anchor=north] {\textsc{yod}};

% Sort the list below with `sort n /{/`

% Phases
% Data collection, formation, theroeticlaelaboration, consolidiation, degeneration
% \phase{Pre-theoretical\\grammar}{650}{780}
    \phase{Precursors}{660}{770}
    \phase{Early grammar}{770}{920} % p.8
    \phase{Codification\\and systematization}{920}{1100}
    \phase{Elaboration of forms\\of presentation}{1100}{1520}


% List of grammarians
    \person{Abū l-Aswad ad-Duʾalī}{688}{}
    \person{\llap{ʿ}Abd Allāh ibn Abī Isḥāq}{734}{} % p.1
    % \person[-1.7ex]{al-Xalil}{786}{Kitāb al-ʿAyn}
    \person{Sībawayhi}{798}{al-Kitāb} % p. 1. In Carter (2004:15) 796
    \person{al-Farrāʾ}{822}{Maʿānī al-Qurʾān} % p.5
    \person[.7ex]{al-Axfaš al-Awsaṭ}{830}{Maʿānī al-Qurʾān} % p.5 835. EI 830
    % \person{al-Aṣmaʿī}{828}{}
    \person[-1.5ex]{al-Mubarrad}{898}{Kitāb al-Muqtaḍab}
    \person{Ṯaʿlab}{904}{} % p.7
    \person{Ibn as-Sarrāj}{928}{Kitāb al-Uṣūl} 
    % \infobox{\textit{First expression of final canonical grammar.}}
    \person{az-Zajjājī}{951}{Kitāb al-Iḍāḥ} % p 11
    \person{as-Sirāfī}{979}{Šarḥ kitāb Sībawayhi} % p.14 in passing
    \person{ar-Rummānī}{994}{Šarḥ kitāb Sībawayhi} % p.14 in passing
    \person[1ex]{Ibn Jinnī}{1002}{al-Xaṣāʾiṣ}
    % \person[-2ex]{Ibn Bābašāḏ}{1077}{}
    \person{al-Jurjānī}{1078}{Dalāʾil al-iʿjāz}
    % \infobox{\textit{with Jurjânî (d.1078) Arabic theory had reached its most sophisticated level} \citep{owens_foundations_1988}}
    \person{Ibn Maḍāʾ}{1196}{ar-Radd ʿalā an-nuḥāt} % p.58 1208. Versteegh (2013:208) 1196
    \person{Abū l-Barakāt al-Anbārī}{1181}{Lumaʿ al-adilla} % p.17
    \person{az-Zamaxšarī}{1143}{al-Mufaṣṣal} % p. 118
    \person{Ibn Yaʿīsh}{1245}{Šarḥ al-Mufaṣṣal}
    % \person{Ibn al-Ḥājib}{1249}{al-Kāfiya}
    % \infobox{\textit{Apogee of pedagogical grammars} \citep{carter_grammatical_2006}}
    \person[-2ex]{Ibn ʿUṣfur}{1271}{al-Mumtiʿ  fī t-taṣrīf} % p.73
    \person{Ibn Mālik}{1274}{al-Alfiyya}
    \person{al-Astarābāḏī}{1287}{Šarḥ al-Kāfiya}
    \person{Ibn Ājurrūm}{1323}{al-Ājurrūmiyya} % IE
    \person{Ibn Hišām}{1359}{Muġnī l-labīb} % p.14
    \person[1ex]{Ibn ʿAqīl}{1367}{Šarḥ al-Alfiyya} % p.14
    \person[-.5ex]{al-Ašmūnī}{1494}{Šarḥ al-Alfiyya} % p.14
    \person{as-Suyūṭī}{1505}{al-Muzhir} % IE
    
    % \dynasty{Muḥammad}{622}{632}
    \dynasty    {OliveGreen}   {Ummayads}      {661}    {750}      {1}
    %\dynasty   {blue}    {ar-Rāšhidun}   {632}    {661}      {1}
    \dynasty    {red}     {Abbasids}      {750}    {1258}     {1}
    \dynasty    {black}   {Fatimids}      {909}    {1171}     {2}
    \dynasty    {blue}     {Ottomans}      {1299}   {\fince}   {1}
    \dynasty    {brown}    {Mamluks}       {1250}   {1517}     {2}

  % Special arrow for Ottomans
  \begin{pgfonlayer}{background}
      % bar
      \draw [ultra thin,yshift=\dynastyvshift, 
          xshift=-.1pt, fill=blue!30!\mixcolor,color=blue!30!\mixcolor]
      (\fince,.5\dynastyheight) 
      -- ++ (.5\dynastyheight,.5\dynastyheight)
      -- ++ (-.5\dynastyheight,.5\dynastyheight) 
      -- cycle;
  \end{pgfonlayer}


	% Axis
	\draw [->, very thick] (\startce,0) -- (\fince,0);

	% Axis ticks
	\foreach \x in {700,800,...,\fince} 
	    {\draw (\x, 0) -- (\x,-.2) node [anchor=north, font=\bfseries] {\x};}

	% Axis tick on 50s
	\foreach \x in {650,675,...,1525} 
	    {\draw (\x, 0) -- (\x,-.1);}

	% Hijra
	\draw [very thick] (622, .2) -- (622, -.5) 
	  node [rotate=-90, anchor=west] {\textbf{\textit{al-Hijra}}~(622)};




\end{tikzpicture}

\vfill\null
\end{document}
