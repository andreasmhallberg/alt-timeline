\documentclass[a4paper,landscape]{article}

\usepackage[top=2cm, bottom=2cm, left=2cm, right=2cm]{geometry}
\usepackage{etoolbox}
\usepackage{graphicx}
\usepackage{ifthen}
\usepackage{footnote} % to save and spew notes
\usepackage{calc}
\usepackage[hidelinks]{hyperref}
\def\UrlFont{\rmfamily\itshape} % roman font in urls
\usepackage{ragged2e}
\usepackage{tikz}
\usepackage{varwidth} % nodes size with adjusted width
\usetikzlibrary{calc, backgrounds}
\usepackage{polyglossia}
\setmainlanguage{english}
\setotherlanguage{arabic}

\usepackage{fontspec}
\setmainfont[Numbers=OldStyle]{Linux Libertine O}
\setsansfont[Numbers=OldStyle]{Source Sans Pro}

% Constants
\def\startce{600}
\def\fince{1550}

\frenchspacing

\tikzstyle{nodedefault} = [
    inner sep=1pt
    , outer sep=0pt
    , rotate = -90
]

%Verticalal position of beginning of name
% \newlw
	
% Thickness of dynasty bar
\newlength\dynastyheight
\setlength\dynastyheight{.4cm}
\newlength\dynastyvshift
\setlength\dynastyvshift{.2cm}

\newlength\dynwidth

% \grammarian[horizontal offset, or `adj`]{<name>}{<yod c.e.>}{<opus magnus>}{<school (B/Q)>}
\newcommand\person[5][0pt]{
    % Name and yod on top
    \ifthenelse{\equal{#1}{adj}}
    {
	\path (name.north west) 
	node [nodedefault, anchor=south west] (name) {#2};
    }
    {
        \path (#3,6.5) ++ (#1,0)
	    node [nodedefault, anchor=west] (name) {#2};
    }

    % Work below
    \path let \p1 = (name) in (\x1,-2\baselineskip)
       node [font=\itshape, rotate=-90, anchor=west] (work) {#4};

    \draw [thin] (name.east) -- (#3,0) ;

    \node at (name.west) [anchor = south, color=black!30] {\MakeLowercase{\textsc{#5}}};

}

\newcommand\infobox[1]{%
  \draw[gray] let \p1 = (work) in node (infobox)
  [color=gray, anchor=north,draw=black, align=left, font=\scriptsize\sffamily]
   at (\x1,-5) {\begin{varwidth}{2.5cm}\RaggedRight#1\end{varwidth}};
  \draw (infobox) -- (work.east);
}


% \dynasty{<color>}{<name>}{<startyear c.e.>}{<endyear c.e.>}{<graphical level (>0)>}
\newcommand\dynasty[5]{%
\begin{pgfonlayer}{background}
    % bar
    \draw [yshift=\dynastyvshift+#5\dynastyheight, line width=\dynastyheight, color=#1!30]
    (#3,0) -- (#4,0);
    % label
    \node [white, yshift=\dynastyvshift+#5\dynastyheight, font=\sffamily] 
    at ({#3+(#4-#3)/2},0) {#2};
\end{pgfonlayer}
}


% phases
% \phase{<name>}{<begin>}{<end>}
\newcommand{\phase}[3]{
\draw [color=black!30, yshift=4mm] (#2+2, 6.8) -- (#2+2, 7) -- (#3-2, 7) node [midway, above, align=center, font=\sffamily] {#1} -- (#3-2, 6.8);
}


\setlength\parindent{0pt}

\begin{document}
\thispagestyle{empty}

\begin{minipage}[t]{.4\textwidth}
{\LARGE Timeline of Arab grammarians\\and their major works}\\[\medskipamount]
\textit{v.1.2\\\today}
\end{minipage}
\hfill
\begin{minipage}[t]{.38\textwidth}
This timeline is based primarily on \textit{The Arabic Linguistic Tradition} by G. Bohas, \mbox{J.-P.} Guillaume, and D. Kouloughli (Georgetown University Press, 2006) and lists grammarians mentioned therein. Years are in the common era.
\end{minipage}


\vfill

\begin{tikzpicture}[x=\textwidth/(\fince-\startce)]

    \node at (622,6.5) [anchor=south, align=center, color=black!30] {\small Baṣran/Kufan};

% Sort the list below with `sort n /{/`

% Phases
% Data collection, formation, theroeticlaelaboration, consolidiation, degeneration
% \phase{Pre-theoretical\\grammar}{650}{780}
    \phase{Precursors}{660}{770}
    \phase{Early grammar}{770}{920} % p.8
    \phase{Codification\\and systematization}{920}{1100}
    \phase{Elaboration of forms\\of presentation}{1100}{1520}


% List of grammarians
    \person{Abū l-Aswad ad-Duʾalī}{688}{}{}
    \person{ʿAbd Allāh ibn Abī Isḥāq}{734}{}{} % p.1
    \person[-1ex]{al-Xalīl}{786}{Kitāb al-ʿAyn}{}
    \person[adj]{Sībawayhi}{798}{al-Kitāb}{B} % p. 1. In Carter (2004:15) 796
    \person[adj]{al-Kisāʾī}{805}{}{K} 
    \person[adj]{al-Farrāʾ}{822}{Maʿānī al-Qurʾān}{K} % p.5
    \person[adj]{al-Axfaš al-Awsaṭ}{830}{Maʿānī al-Qurʾān}{B} % p.5 835. EI 830
    % \person{al-Aṣmaʿī}{828}{}
    \person[adj]{Ibn as-Sikkīt}{850}{}{K}
    \person[-1ex]{al-Mubarrad}{898}{Kitāb al-Muqtaḍab}{B}
    \person[adj]{Ṯaʿlab}{904}{}{K} % p.7
    \person{Ibn as-Sarrāj}{928}{Kitāb al-Uṣūl}{B}
    % \infobox{\textit{First expression of final canonical grammar.}}
    \person{az-Zajjājī}{951}{Kitāb al-Iḍāḥ}{B} % p 11
    \person{as-Sirāfī}{979}{Šarḥ kitāb Sībawayhi}{B} % p.14 in passing
    \person{ar-Rummānī}{994}{Šarḥ kitāb Sībawayhi}{} % p.14 in passing
    \person[adj]{Ibn Jinnī}{1002}{al-Xaṣāʾiṣ}{}
    % \person[-2ex]{Ibn Bābašāḏ}{1077}{}
    \person{al-Jurjānī}{1078}{Dalāʾil al-iʿjāz}{}
    % \infobox{\textit{with Jurjânî (d.1078) Arabic theory had reached its most sophisticated level} \citep{owens_foundations_1988}}
    \person{Ibn Maḍāʾ}{1196}{ar-Radd ʿalā an-nuḥāt}{} % p.58 1208. Versteegh (2013:208) 1196
    \person{Abū l-Barakāt al-Anbārī}{1181}{Lumaʿ al-adilla}{} % p.17
    \person{az-Zamaxšarī}{1143}{al-Mufaṣṣal}{} % p. 118
    \person{Ibn Yaʿīsh}{1245}{Šarḥ al-Mufaṣṣal}{}
    % \person{Ibn al-Ḥājib}{1249}{al-Kāfiya}
    % \infobox{\textit{Apogee of pedagogical grammars} \citep{carter_grammatical_2006}}
    \person[-1.5ex]{Ibn ʿUṣfur}{1271}{Mumtiʿ  fī t-taṣrīf}{} % p.73
    \person[adj]{Ibn Mālik}{1274}{al-Alfiyya}{}
    \person[adj]{al-Astarābāḏī}{1287}{Šarḥ al-Kāfiya}{}
    \person{Ibn Ājurrūm}{1323}{al-Ājurrūmiyya}{} % IE
    \person{Ibn Hišām}{1359}{Muġnī l-labīb}{} % p.14
    \person[adj]{Ibn ʿAqīl}{1367}{Šarḥ al-Alfiyya}{} % p.14
    \person{al-Ašmūnī}{1494}{Šarḥ al-Alfiyya}{} % p.14
    \person[adj]{as-Suyūṭī}{1505}{al-Muzhir}{} % IE
    
    % \dynasty{Muḥammad}{622}{632}
    \dynasty    {green}   {Ummayads}      {661}    {750}      {1}
    %\dynasty   {blue}    {ar-Rāšhidun}   {632}    {661}      {1}
    \dynasty    {red}     {Abbasids}      {750}    {1258}     {1}
    \dynasty    {black}   {Fatimids}      {909}    {1171}     {2}
    \dynasty    {blue}     {Ottomans}      {1299}   {\fince}   {1}
    \dynasty    {brown}    {Mamluks}       {1250}   {1517}     {2}

  % Special arrow for Ottomans
  \begin{pgfonlayer}{background}
      % bar
      \draw [ultra thin,yshift=\dynastyvshift, 
          xshift=-.1pt, fill=blue!30,color=blue!30]
      (\fince,.5\dynastyheight) 
      -- ++ (.5\dynastyheight,.5\dynastyheight)
      -- ++ (-.5\dynastyheight,.5\dynastyheight) 
      -- cycle;
  \end{pgfonlayer}


	% Axis
	\draw [->, very thick] (\startce,0) -- (\fince,0);

	% Axis ticks
	\foreach \x in {700,800,...,\fince} 
	    {\draw (\x, 0) -- (\x,-.2) node [anchor=north, font=\bfseries] {\x};}

	% Hijra
	\draw [very thick] (622, .2) -- (622, -.5) 
	  node [rotate=-90, anchor=west] {\textbf{\textit{al-Hijra}}~(622)};




\end{tikzpicture}

\vfill\null
\center
\begin{minipage}{8cm}
\center
\tiny
Written by Andras Hallberg.
This document is licensed under a Creative Commons Attribution 4.0 International license. 
It is available for download at \url{http://andreasmhallberg.github.io/images/alt-timeline.pdf}.
For suggestions on improvements, contact the author at \href{mailto:andreasmartenhallberg@gmail.com}{\itshape andreasmartenhallberg@gmail.com}.
\end{minipage}
\vspace{-1cm}
\end{document}

